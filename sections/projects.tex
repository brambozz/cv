
\cvsection{Notable projects}

\cvsubsection{Detection of Adhesions on Cine-MRI with Spatio-Temporal Deep Learning}

\vspace{-1mm}
\paragraphstyle{
Post-surgical adhesions can cause pain by connecting tissue and organs in the abdominal cavity.
Cine-MRI is a non-invasive modality which captures dynamic motion during deep breathing, revealing adhesion presence through abnormal motion patterns.
Improving adhesion diagnosis by development of AI could help bring this modality to more hospitals and help patients worldwide suffering from chronic pain.
This unique medical AI challenge requires computer vision models to reason about both spatial and temporal features in order to detect tissue structures that stick together.
We have investigated this problem from multiple perspectives, starting with characterizing observer variability (publication \ref{paper:inter}) and motion quality (publication \ref{paper:quantifiable}).
In terms of AI development, we experimented with recurrent neural networks (publication \ref{paper:midl}) and deep learning segmentation approaches using manual annotations and a U-Net framework, currently under review.
Currently, another observer study is in draft, where radiologists are assisted by deep learning models.
Throughout the project, we have shown that adhesion detection on cine-MRI is a particularly hard challenge, and we have presented the first fully end-to-end deep learning approaches for this problem.
This project was my primary focus in the PhD, and consequently I led the project from hypothesis, data collection, technical development, to dissemination of the results.
}

\cvsubsection{Adapting Latent Diffusion Models to Medical Imaging}

\vspace{-1mm}
Latent diffusion models are popular and highly succesful in (text-to-)image generation. Generative imaging can be highly relevant for the medical domain, especially for data-scarce settings like rare diseases. We explored if pre-trained latent diffusion models (Stable Diffusion) can be adapted to the medical domain. In publication \ref{paper:diffusion}, we demonstrate that textual inversion allows diagnostically relevant image generation for prostate MRI, chest X-ray and histopathology, using only 100 examples on a consumer-grade GPU. I performed the main experiments and wrote the paper.

\cvsubsection{Uncertainty-Guided Semi-Supervised Segmentation}

\vspace{-1mm}
The MICCAI2022 Fast and Low-resource Semi-Supervised Abdominal Organ Segmentation (FLARE) challenge investigated two typically lacking factors in medical imaging: annotations and compute.
The challenge task was to segment 13 organs on CT scans, provided 50 fully annotated cases and 2000 unannotated cases, while minimizing computational resources.
With a colleague PhD, we developed a simple uncertainty-guided framework, where we use predictive uncertainty estimated via model ensembles to judge the quality of pseudo-labels generated for the unlabeled cases.
We found that using uncertainty to filter out low-quality pseudo-labels leads to improved segmentation performance compared to not filtering at all (publications \ref{paper:uncertainty}, \ref{paper:flare}).
A journal paper is currently under review, where we compare uncertainty metrics in detail to multiple baselines and found that metrics tailored towards the clinical task are more informative.
We shared most work equally, with emphasis an on code and experiments for me.

\cvsubsection{MSc thesis: Hopping Transport in Disordered Dopant Networks}

\vspace{-1mm}
Human brains operate at a fraction of the energy requirements of today’s AI systems, and conventional 2D silicon hardware is part of the reason.
Neuromorphic hardware researchers investigate novel computational substrates which can address this shortcoming.
I investigated numerical simulations of dopant networks in silicon, which allow for efficient exploration of the algorithmic capabilities of such systems without the need for physical samples (publication \ref{paper:hopping}).
These dopant networks are a promising new hardware substrate for energy-efficient intelligent systems, which we demonstrated by classifying digits \textit{in materio} (publication \ref{paper:nature}).
I formalized and initiated the development of the numerical simulation software. Additionally, I was involved in maintaining the lab's measurement and data collection software in Python.
